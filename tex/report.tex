\documentclass{DIKU-article}[2010/01/13]

\usepackage[utf8]{inputenc}
\usepackage[OT1]{fontenc}
\usepackage[danish]{babel}
\usepackage{t1enc}

%\selectdanish % Add this if your report is in Danish

\usepackage{epsfig} 

%\setlength{\errorcontextlines}{999} %Can be used for debugging purposes
%\alternativecitationstyle
%\draft

\titlehead{Self-adjusting heaps}
\authorhead{Johan Brinch and Asser Schrøder Femø}

\title{Self-adjusting heaps: A performance comparison}

\author{
Johan Brinch
\and
Asser Schrøder Femø
}

\institute{
Department of Computing, University of Copenhagen\\
Universitetsparken 1, DK-2100 Copenhagen East, Denmark\\
\email{...@diku.dk}
\and
\email{asser@diku.dk}
}

\dates{CPH STL Report 2010-?, January 2010.}

\begin{document}

\maketitle

\begin{abstract}
\end{abstract}

\begin{subject}
\end{subject}

\section{Introduction}

The aim of this project is to develop a meldable priority queue framework in
order to implement and compare the following pairing heap variants: The original
pairing heap by Fredman et al, the pairing heap with Stasko-Vitter lazy insertion and
the pairing heap with costless meld by Elmasry. We also include performance
comparisons with existing CPH STL priority queues.

\section{Priority queue variants}

\begin{description}
\item{Original pairing heap} by Fredman et al\cite{fredman} is intended as an
alternative to the Fibonacci heap. While theoretically slower (conjectured just
as fast but not proven) it is much simpler to implement and faster in practice.
All operations run in $O(\log n)$ amortized time.

\item{Stasky-Vitter lazy insertion} expands on the original pairing heap by
modifying the insert procedure to maintain an auxiliary list, or forest, of
nodes (or single-node trees) to be inserted into the heap tree which is then
merged into the main tree when a delete-min operation is executed. This
modification improves the bound for insert to $O(1)$ amortized.

\item{Pairing heap with Costless Meld} by Elmasry\cite{elmasry} expands on
the original heap by having decrease-key also use an auxiliary list to
determine whether a nodes key was decreased deferring the clean-up work to a
later time. This makes decrease-key run in $O(\log \log n)$ amortized time.
Melding two heaps only cleans up the auxiliary structures of the smaller heap
before pairing the heaps as in the original method, making meld have zero
amortized cost.

\item{Pairing heap with Costless Meld and Lazy Insert} is a combination of the
two abovementioned heaps, with both extensions in use.
\end{description}

\section{Implementation}

We have implemented all three variants into one pairing queue framework. Due to
the small differences between the variants we use a policy template to determine
which variant to use. 

Our framework is based on the code for Priority-queue-framework in CPH STL as it
shares the interface and the same general structure. We considered the
possibility of simply extending or directly using this framework but due to
differences in the internal node structure we decided against that.
Specifically, Priority-queue-framework's \verb!heap_node! structure allows only
parent, left and right pointers where we need a sibling list structure because
pairing heaps use multiway trees.

\subsection{Main framework}

The framework consists mainly of the class \verb!pairing_heap_framework! which
is based on \verb!cphstl::priority_queue_framework!. It implements
some of the \verb!extract! logic that is identical for all our pairing heap
variants. For some cases, and for all other heap operations it defers to the
\emph{policy}, which is given as template parameter \verb!P!.\\
\\
\verb!pairing_heap_node! contains the pointers neede for the child, sibling
structure (a left, right and parent pointer). An obvious modification would be
to extend the node structure and methods to be able to contain only two pointers
as described in the original pairing heap paper\cite{fredman}. A
\verb!pairing_heap_node! is able to meld itself with another tree (assuming both
are root nodes) as well as into and out of an auxiliary list.

\subsection{Policies}

General for the policies is, as they do almost all the modifications of the heap
tree and/or auxiliary lists there is a lot of pointer juggling; it is easy to
mess up the structure of the heap. Therefore we have debug methods in place to
ensure the validity of the data structure (the \verb!is_valid! and
\verb!is_valid_tree! methods), which proved very helpful during implementation
and bugfixing.

\subsubsection{Strict policy}

This implements the original pairing heap.

\subsubsection{Lazy insert policy}

The lazy insert policy uses \verb!cphstl::doubly_linked_list! as a container for
the auxiliary list, or insertion buffer.

\subsubsection{Lazy increase policy}

This policy implements the costless meld, or lazy increase heap.

In the costless meld heap, when doing phase 2 of the cleanup we can save a few
comparisons by observing that when combining the $\Theta(\log n)$ trees in each
group we already know the order in which to link the trees since they have been
sorted according to the root nodes. Our link operation can then just link the
trees without doing the usual comparison between the root nodes. While this is
not an actual saving (we just moved the comparison from the link operation to
the sorting of the trees) it is an important implementation detail easily
overlooked.

\subsection{Usage with CPH STL}

We have successfully managed to adhere to the meldable priority queue interface
of CPH STL (\verb!cphstl::stl-meldable-priority-queue!). Our priority queue
framework can be given as the realizator template argument and should work
without further modifications, which has been verified by running the benchmarks
in \verb!Source/Priority-queue-framework/Benchmark!.

Working with and understanding CPH STL proved somewhat difficult and while we
were able to quickly make an implementation of the pairing heaps, it took a
while to integrate everything and weed out the bugs. Our lacking experience with
C++ and templating in particular probably didn't help.

- Benchmarking tests
- Difficulties with different versions of Priority-queue-framework,
  Meldable-priority-queue and Iterator from CPH STL. Code was updated while the
  project was in progress.
- Memory usage .. three pointers per node?
- Usage of other CPH STL code .. cphstl::doubly\_linked\_list for example

\section{Performance}

In order to run the code you need 2010 versions of Iterator and
Meldable-priority-queue from CPH STL (anything after 2010-01-01 should be fine),
but pre-2010 Priority-queue-framework files.

Additionally we had to fix a bug in Iterator/Code/priority-queue-iterator.h++
line 77; change template parameter to F (or something else that isn't used
elsewhere in the file) since E was already declared .

- Sorting of the auxiliary forest in costless meld takes a long time. Could we
  optimize that somehow?

- Memory usage? Three pointers per node vs. two pointers per node?

\begin{acknowledgements}
This section comes before the References and is unnumbered.
\LaTeX-en\-viron\-ment is \verb|acknowledgements|.
\end{acknowledgements}

\bibliographystyle{DIKU} % Use DIKU-alternative for the other citation style
\bibliography{report}

\end{document}

