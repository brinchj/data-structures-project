\documentclass{DIKU-article}[2010/01/13]

\usepackage[latin1]{inputenc}
\usepackage[OT1]{fontenc}

%\selectdanish % Add this if your report is in Danish

\usepackage{epsfig} 

%\setlength{\errorcontextlines}{999} %Can be used for debugging purposes
%\alternativecitationstyle
%\draft

\titlehead{Self-adjusting heaps}
\authorhead{Johan Brinch and Asser Schroder Femo}

\title{Self-adjusting heaps: A performance comparison}

\author{%
Johan Brinch
\and
Asser Schroder Femo
}

\institute{%
Department of Computing, University of Copenhagen\\
Universitetsparken 1, DK-2100 Copenhagen East, Denmark\\
\email{...@diku.dk}
\and
\email{asser@diku.dk}%
}

\dates{CPH STL Report 2010-?, January 2010.}

\begin{document}

\maketitle

\begin{abstract}
\end{abstract}

\begin{subject}
\end{subject}

\section{Introduction}

The aim of this project is to develop a meldable priority queue framework in
order to implement and compare the following pairing heap variants: The original
pairing heap by Fredman et al, the pairing heap with Stasko-Vitter lazy insertion and
the pairing heap with costless meld by Elmasry. We also include performance
comparisons with existing CPH STL priority queues.

\section{Priority queue variants}

\begin{description}
\item{Original pairing heap} by Fredman et al\cite{fredman} is intended as an
alternative to the Fibonacci heap. While theoretically slower it is much simpler 
to implement and faster in practice. All operations run in $O(\lg n)$ amortized time.

\item{Stasky-Vitter lazy insertion} expands on the original pairing heap by
modifying the insert procedure to maintain an auxiliary list, or forest, of
nodes to be inserted into the heap tree which is then merged into the main tree
when a delete-min operation is executed. This modification improves the bound
for insert to $O(1)$ amortized.

\item{Pairing heap with Costless meld} by Elmasry\cite{costlessmeld} expands on
the lazy insertion method by having decrease-key also use an auxiliary list to
determine whether a nodes key was decreased deferring the clean-up work to a
later time. This makes decrease-key run in $O(\lg \lg n)$ amortized time.
Melding two heaps only cleans up the auxiliary structures of the smaller heap
before pairing the heaps as in the original method, making meld have zero
amortized cost.
\end{description}

\section{Implementation}

We have implemented all three variants into one pairing queue framework. Due to
the small differences between the variants we use a policy template to determine
which variant to use. 

- Framework
- Policies
- Integration with stl-meldable-priority-queue as realizator
- Benchmarking tests
- Difficulties with different versions of Priority-queue-framework,
  Meldable-priority-queue and Iterator from CPH STL. Code was updated while the
  project was in progress.

\section{Performance}

In order to run the code you need 2010 versions of Iterator and
Meldable-priority-queue, but pre-2010 Priority-queue-framework files.

\begin{acknowledgements}
This section comes before the References and is unnumbered.
\LaTeX-en\-viron\-ment is \verb|acknowledgements|.
\end{acknowledgements}

\bibliographystyle{DIKU} % Use DIKU-alternative for the other citation style
\bibliography{report}

\end{document}

