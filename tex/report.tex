\documentclass{DIKU-article}[2010/01/13]

\usepackage{palatino}
\usepackage[utf8]{inputenc}
\usepackage[OT1]{fontenc}
\usepackage[danish]{babel}
\usepackage{t1enc}
\usepackage{fancyref}
\usepackage{listings}

\lstset{breaklines=true, language=c++, basicstyle=\footnotesize}

%\selectdanish % Add this if your report is in Danish

\usepackage{epsfig}

\newcommand{\code}[1]{{\small\texttt{#1}}}
\usepackage{microtype}

%\setlength{\errorcontextlines}{999}
%Can be used for debugging purposes
%\alternativecitationstyle
%\draft

\titlehead{Self-adjusting heaps}
\authorhead{Johan Brinch and Asser Schrøder Femø}

\title{Self-adjusting heaps: A performance comparison}

\author{
Johan Brinch
\and
Asser Schrøder Femø
}

\institute{
Department of Computing, University of Copenhagen\\
Universitetsparken 1, DK-2100 Copenhagen East, Denmark\\
\email{zerrez@diku.dk}
\and
\email{asser@diku.dk}
}

\dates{CPH STL Report 2010-\#, January 2010.}

\begin{document}

\maketitle

%\begin{abstract}
%\end{abstract}

%\begin{subject}
%\end{subject}

\section{Introduction}

The aim of this project is to develop a meldable priority queue
framework in order to implement and compare the following pairing heap
variants:

\begin{enumerate}
\item Strict Pairing Heap: The original pairing heap by Fredman et
  al\cite{fredman}
\item Lazy-Insert Pairing Heap: The strict pairing heap using
  Stasko-Vitter's lazy insertion\cite{stasko}
\item Costless-Meld Pairing Heap: The pairing heap with costless meld
  by Elmasry\cite{elmasry}.
\end{enumerate}

The new pairing heap framework will work as a realizator with the
current meldable priority queue framework already available in the
Copenhagen Standard Library for C++ (CPH STL). As such, it will be
easy to switch between the three implemented data structures.

Furthermore, we wish to compare the performance of these priority
queue variations against each other and against the current
implementation of priority queues in the CPH STL.

We expect that the reader is familiar with the CPH STL and it's
priority queue frameworks\cite{cphstl-framework}. Furthermore, the
reader should be familiar with the aspects of amortized running time
analysis.



\section{Terminology}
In this paper, we use the term "`max heap"' of the newly implemented
heaps, since the CPH STL refers to max heaps in the implementation (by
having \code{increase}). Of course, the implementation supports any
comparison operator. The newly implemented priority queues are
referred to as pairing heaps while the existing are called priority
queues (though the words mean the same, this is what their respective
implementations call them).

We denote the original pairing heap\cite{fredman} a Strict Pairing
Heap. The Stasko-Vitter heap\cite{stasko} a Lazy-Insertion Pairing
Heap. And finally the Elmasry\cite{elmasry} heap a Costless-Meld
Pairing Heap.

We name the operations of a max heap as:
\begin{description}
\item{\code{insert}}: Insertion of a new node
\item{\code{extract-max}}: Extract the current maximum
\item{\code{extract}}: Extract a specific node other than maximum
\item{\code{increase}}: Increase the priority of a node
\item{\code{meld}}: Meld two max heaps
\item{\code{maximum}}: Return the node that is the current maximum
\end{description}

We use the term "`tree"' for a tree having the heap-condition. We
use the term "`node"' when referring to either a tree of one node or
a node inside a tree. In general, nodes are sub-trees.

\section{Priority queue variants}
\subsection{Overview}
\begin{description}
\item{Strict Pairing Heap} by Fredman et al\cite{fredman} was
  developed as an alternative to the Fibonacci heap. While
  theoretically slower (conjectured just as fast but not proven) it is
  much simpler to implement and seems to be faster in practice.

  Beneath is a summary of it's operations:
  \begin{description}
  \item{\code{insert}}: Create a one-node heap consisting of the new
    element and \code{meld} the two heaps.
  \item{\code{extract-max}}: Cut the root node out of the tree. Meld
    the child nodes using a two-pass melding operation. The new root
    is the root of the resulting tree.
  \item{\code{extract}}: Cut the element to extract. Meld the child
    nodes using a two-pass melding operation. Meld the main tree with
    the resulting tree.
  \item{\code{increase}}: If the node to increase the root just
    increase return. If not, \code{extract} the node, increase it and
    \code{insert} it.
  \item{\code{meld}}: Compare the two root nodes and make the smaller a
    child of the larger.
  \end{description}
  During these operations a pointer to the maximum of the heap is
  maintained. As such, the \code{maximum} operation consists of simply
  returning this pointer.\\

\item{Lazy-Insert Pairing Heap} by Stasko and Vitter expands on the
  Strict Pairing Heap by modifying the \code{insert} operation to be
  lazy. Instead of inserting the new element right away it is added to
  an auxiliary list, or forest, of nodes (or single-node trees) to be
  inserted into the heap tree. On the following \code{extract-max}
  operation this list is emptied and it's waiting nodes melded using a
  multi-pass strategy. Hereafter the resulting tree is melded with the
  main tree. The motivation is to pack together new nodes to avoid too
  many root-children.

\item{Pairing heap with Costless Meld} by Elmasry\cite{elmasry}
  expands on the Strict Pairing Heap by modifying the \code{decrease}
  operation to use an auxiliary list. This list keeps track of which
  nodes has been decreased during operation. The actual
  \code{decrease} operation is performed by a cleanup method during
  the following \code{extract-max}. Furthermore, the \code{meld}
  operation is modified to cleanup the smaller heap before melding.

  The \code{cleanup} operation works as follows:
  \begin{enumerate}
  \item Group the nodes in the increase-list in groups of size $\log
    n$
  \item Sort the nodes in each group in increasing order
  \item Meld the nodes in each group, the result being several unary
    trees
  \item Meld the unary trees to the main tree
  \end{enumerate}

%\item{Pairing heap with Costless Meld and Lazy Insert} is a
%  combination of the two abovementioned heaps, with both extensions in
%  use.
\end{description}

\subsection{Amortized Running Time}
\Fref{tab:running_time} shows the amortized running time for each
data structure.
\begin{table}[h]
\label{tab:running_time}
\caption{Amortized Running Times}
\begin{tabular}{l|lllll}
 & \code{insert} & \code{extract-max}  &
 \code{increase} & \code{meld} \\ \hline
Strict & $O(\log n)$ & $O(\log n)$ & $O(\log n)$ & $O(\log n)$ \\
Lazy-Insert & $O(1)$ & $O(\log n)$ & $O(\log n)$ & $O(\log n)$ \\
Costless-Meld & $O(1)$ & $O(\log n)$ & $O(\log \log n)$ & $zero$ \\
\end{tabular}
\end{table}

The theoretical amortized running time is best for the Costless-Meld
variant. A goal of this project is to investigate whether this is true
in practical applications.

\section{Implementation}

We have implemented a pairing heap framework that generates meldable
queue realizators. We have implemented the three discussed pairing
heap variations as policies for this framework.

Our framework is based on the code for Priority-queue-framework in CPH
STL as it shares the interface and the same general structure. We
considered the possibility of simply extending or directly using this
framework but due to differences in the internal node structure we
decided against this.  Specifically, Priority-queue-framework's
\code{heap\_node} structure allows only \code{parent}, \code{left} and
\code{right} pointers where we \code{left}, \code{right} and
\code{child} pointers because the pairing heaps are based on multiway
trees. Furthermore, we need the possibility of marking a node.

\subsection{Main framework}

The framework consists mainly of the class:\\
\indent \code{cphstl::}\code{pairing\_heap\_framework}

\noindent which is based on the existing class:\\
\indent \code{cphstl::priority\_queue\_framework}\\

The new class functions mainly as an interface, but also implements
some of the logic that is shared amongst all heap variations. This
includes updating the current \code{size} of the heap and handling
trivial cases of the \code{extract} operations. For non-trivial
cases, and for all other heap operations, it defers to the
\emph{policy}, which is given as the template parameter \code{P}.


\code{pairing\_heap\_node} contains the pointers needed for the child
and sibling structure (a left, right and parent pointer). It also
contains a colour, to allow for marking of nodes, and a
\code{list\_node} pointer that is only used by the Lazy-Increase
variant.
Nodes contains the logic needed to \code{add} and \code{cut} a
node from a double-linked list as well as to \code{cut} a node or
\code{swap} one node for another in a tree. Two nodes can also be
melded together, forming a new tree with the largest node as root and
the other as it's first child.

An improvement to this structure would be to take an extra template
parameter for a structure containing other needed information, such as
for example colour and \code{list\_node} pointers. However, since the
implementation itself is not the main focus here, we decided to leave
this out for later improvements.

A possible modification would be to change the node structure and
methods to be able to contain only two pointers as described in the
original paper on Strict Pairing Heaps\cite{fredman}.

\subsection{Policies}

A policy describes a pairing heap variant. Specifically, it specifies
the logic needed to:
\begin{itemize}
\item \code{insert} when the heap is non-empty,
\item \code{extract} when the node is not maximum.
\item \code{extract-max} when the heap contains more then 2 nodes or the
  requested node is not root.
\item \code{increase}
\item \code{meld}
\end{itemize}

General for the policies is, that they involve a of pointer
juggling. Although we have tried to hide some of the pointer
operations inside the \code{heap\_node} structure, most of the
operations performed by the policies are still just that.

Because of this, it is easy to loose the overview of the
structure. This makes debugging difficult. To aid this we have
implemented debug some methods that (if run) ensure the validity of
the data structure (specifically the \code{is\_valid} and
\code{is\_valid\_tree} methods), which proved very helpful during
implementation and bug fixing.

\subsubsection{Strict Policy}

This policy implements the Strict Pairing Heap. It uses no extra
containers or pointers than what is provided in \code{heap\_node}.

\subsubsection{Lazy-Insert Policy}
\label{sec:policy-lazy-insert}

This policy implements the Lazy-Insert Pairing Heap. The lazy insert
policy uses an extra doubly-linked list, which is implemented using
the existing \code{left} and \code{right} pointers in the
\code{heap\_node} structure. This is possible, since each node exists
\emph{either} in the tree or in the list. The \code{color\_} variable
in \code{heap\_node} is also used to keep track of which nodes are
currently in the list and which are not. This prevents multiple
insertions of the same node.

\subsubsection{Lazy-Increase Policy}

This policy implements the Costless-Meld Pairing Heap. The
costless-meld policy uses an extra \code{cphstl::doubly-linked-list}
to keep track of the nodes that have been increased. We cannot reuse
the sibling pointers in the \code{heap\_node} structure here, since
nodes in the increase-list remains inside the tree.

As with the lazy-insert policy in \fref{sec:policy-lazy-insert}, the
\code{color\_} variable of the \code{heap\_node} is used to avoid
multiple insertions of the same nodes. To allow for efficient
\code{extract} operations, each node maintains a \code{node\_in\_list}
pointer, which is updated if the node is added to the
increase-list. An improvement to this implementation is to eliminate
the \code{color\_} variable in this policy and only use the
\code{node\_in\_list} pointer. However, this is not currently
possible, since the implementation does not allow for arbitrary
satellite structures.

In the costless-meld heap, when doing phase 2 of the cleanup we can
save a few comparisons by observing that when combining the
$\Theta(\log n)$ trees in each group we already know the order in
which to link the trees since they have been sorted according to the
root nodes. Our link operation can then just link the trees without
doing the usual comparison between the root nodes. While this is not
an actual saving (we just moved the comparison from the link operation
to the sorting of the trees) it is an important implementation detail
easily overlooked.

\subsection{Usage with CPH STL}

We have successfully managed to adhere to the meldable priority queue
interface of CPH STL. Our priority queue framework can be given as the
realizator template parameter and should work without further
modifications, which has been verified by running the benchmarks
belonging to the CPH STL priority queue framework\footnote{See the
  path: \code{Source/Priority-queue-framework/Benchmark}}.

Our meldable priority queues can thereby integrate with existing C++
code that depends on the CPH STL queues.

One important remark, is that the CPH STL undertook some serious
changes while we wrote our code. These changes were related to the
priority queue implementation and its benchmark programs. To avoid too
many complications (the CPH STL changing as our implementation
evolved) we chose to base our code on the revision from
2009-12-24. However, this version of the CPH STL only works if the
Meldable-priority-queue-framework and the Iterator modules are checked
out from newer versions. Respectively from 2010-01-06 and 2010-01-13.

Additionally we had to fix a bug in the Iterator module in the file:\\
\indent \code{Iterator/Code/priority-queue-iterator.h++} (line 77)

We had to change the template parameter E to F (or something else that
is not already used elsewhere in the file) since E was already
declared.


\section{Benchmark}

We have run the standard benchmarks from the CPH STL module.

\subsection{Comparisons}
\Fref{tab:comp.first} through \fref{tab:comp.last} present our results
on comparison counts per operation. In general the three implemented
method performs as good or better than the pennant queue and weak
queue from the CPH STL. There are few variations between our three
variants and all the numbers seems to be of the same order.

\subsection{CPU Timings}
\Fref{tab:time.first} through \fref{tab:time.last} present our results
on cpu time per operation. All numbers listed are in seconds. In
general the three implemented method performs as good or better than
the pennant queue and weak queue from the CPH STL.


\begin{acknowledgements}
We would like to thank DIKU Kantinen$^{\mathtt{TM}}$ for keeping us
warm on the cold winther nights and for their infinite supply of coffee.
\end{acknowledgements}


\section{Conclusion}

Working with and understanding CPH STL proved somewhat difficult and
while we were able to quickly make an implementation of the pairing
heaps, it took a while to integrate everything and weed out the
bugs. Our lacking experience with C++ and templating in particular
took up a great part of our time.

\bibliographystyle{DIKU}
\bibliography{report}


\newpage
\appendix


\section{Tables}
\begin{table}[h!]
\centering
\caption{Comp. Push}
\begin{tabular}{l|lll}
\label{tab:comp.first}
 & 10.000 & 100.000 & 500.000 \\
\hline
Strict & $1$ & $1$ & $1$ \\
Lazy-Insert & $1$ & $1$ & $1$ \\
Lazy-Increase & $1$ & $1$ & $1$ \\
Pennant & $3.29$ & $3.29$ & $3.29$ \\
Weak & $2$ & $2$ & $2$

\end{tabular}
\end{table}

\begin{table}[h!]
\centering
\caption{Comp. Increase}
\begin{tabular}{l|lll}
 & 10.000 & 100.000 & 500.000 \\
\hline
Strict & $1$ & $1$ & $1$ \\
Lazy-Insert & $1$ & $1$ & $1$ \\
Lazy-Increase & $1$ & $1$ & $1$ \\
Pennant & $4.07$ & $4.07$ & $4.08$ \\
Weak & $3.40$ & $3.40$ & $3.40$

\end{tabular}
\end{table}

\begin{table}[h!]
\centering
\caption{Comp. Erase}
\begin{tabular}{l|lll}
 & 10'000 & 100'000 & 500'000 \\
\hline
Strict & $10^{-5}$ & $10^{-6}$ & $2\cdot 10^{-7}$ \\
Lazy-Insert & $10^{-5}$ & $10^{-6}$ & $2\cdot 10^{-7}$ \\
Lazy-Increase & $10^{-5}$ & $10^{-6}$ & $2\cdot 10^{-7}$ \\
Pennant & $2.93$ & $2.91$ & $2.91$ \\
Weak & $2.02$ & $2.03$ & $2.04$

\end{tabular}
\end{table}

\begin{table}[h!]
\centering
\caption{Comp. Pop}
\begin{tabular}{l|lll}
 & 10'000 & 100'000 & 500'000 \\
\hline
Strict & $21.54$ & $25.68$ & $28.56$ \\
Lazy-Insert & $19.58$ & $23.69$ & $26.56$ \\
Lazy-Increase & $21.54$ & $25.68$ & $28.56$ \\
Pennant & $24.35$ & $29.77$ & $33.04$ \\
Weak & $23.32$ & $28.76$ & $32.03$

\end{tabular}
\end{table}

\begin{table}[h!]
\centering
\caption{Comp. Increase 2}
\begin{tabular}{l|lll}
& 10'000 & 100'000 & 500'000 \\
\hline
Strict & $16.28$ & $20.43$ & $23.31$ \\
Lazy-Insert & $16.35$ & $20.44$ & $23.32$ \\
Lazy-Increase & $15.19$ & $19.19$ & $21.99$ \\
Pennant & $25.30$ & $33.54$ & $35.02$ \\
Weak & $23.65$ & $29.43$ & $32.12$

\end{tabular}
\end{table}

\begin{table}[h!]
\centering
\caption{Comp. Increase 3}
\begin{tabular}{l|lll}
\label{tab:comp.last}
& 10'000 & 100'000 & 500'000 \\
\hline
Strict & $1.2$ & $1.2$ & $1.2$ \\
Lazy-Insert & $3$ & $3$ & $3$ \\
Lazy-Increase & $2.2$ & $2.21$ & $2.23$ \\
Pennant & $4.55$ & $4.54$ & $4.54$ \\
Weak & $3.29$ & $3.29$ & $3.29$


\end{tabular}
\end{table}


\begin{table}[h!]
\centering
\caption{Time Push}
\begin{tabular}{l|lll}
\label{tab:time.first}
  & 10.000 & 100.000 & 500.000 \\
  \hline
  Strict & $10^{-7}$ & $1.3\cdot 10^{-7}$ & $1.28\cdot 10^{-7}$\\
  Lazy-Insert & $10^{-7}$ & $1.25\cdot 10^{-7}$ & $1.21\cdot 10^{-7}$\\
  Lazy-Increase & $10^{-7}$ & $1.35\cdot 10^{-7}$ & $1.28\cdot 10^{-7}$\\
  Pennant & $10^{-7}$ & $1.65\cdot 10^{-7}$ & $1.64\cdot 10^{-7}$\\
  Weak & $2\cdot 10^{-7}$ & $1.5\cdot 10^{-7}$ & $1.47\cdot 10^{-7}$

\end{tabular}
\end{table}

\begin{table}[h!]
\centering
\caption{Time Increase}
\begin{tabular}{l|lll}
 & 10.000 & 100.000 & 500.000 \\
\hline
Strict & $2\cdot 10^{-7}$ & $4\cdot 10^{-8}$ & $3.6\cdot 10^{-8}$\\
Lazy-Insert & $0$ & $6\cdot 10^{-8}$ & $2.6\cdot 10^{-8}$\\
Lazy-Increase & $2\cdot 10^{-7}$ & $1.2\cdot 10^{-7}$ & $1.08\cdot 10^{-7}$\\
Pennant & $4\cdot 10^{-7}$ & $2.1\cdot 10^{-7}$ & $2.4\cdot 10^{-7}$\\
Weak & $2\cdot 10^{-7}$ & $2.3\cdot 10^{-7}$ & $2.66\cdot 10^{-7}$

\end{tabular}
\end{table}

\begin{table}[h!]
\centering
\caption{Time Erase}
\begin{tabular}{l|lll}
 & 10'000 & 100'000 & 500'000 \\
\hline
Strict & $2\cdot 10^{-7}$ & $2\cdot 10^{-8}$ & $8.6\cdot 10^{-8}$\\
Lazy-Insert & $0$ & $7\cdot 10^{-8}$ & $8.6\cdot 10^{-8}$\\
Lazy-Increase & $10^{-7}$ & $1.1\cdot 10^{-7}$ & $8.8\cdot 10^{-8}$\\
Pennant & $4\cdot 10^{-7}$ & $3.8\cdot 10^{-7}$ & $3.72\cdot 10^{-7}$\\
Weak & $4\cdot 10^{-7}$ & $3.4\cdot 10^{-7}$ & $3.42\cdot 10^{-7}$

\end{tabular}
\end{table}

\begin{table}[h!]
\centering
\caption{Time Pop}
\begin{tabular}{l|lll}
  & 10'000 & 100'000 & 500'000 \\
  \hline
  Strict & $1.5\cdot 10^{-6}$ & $2.06\cdot 10^{-6}$ & $3.27\cdot 10^{-6}$\\
  Lazy-Insert & $1.2\cdot 10^{-6}$ & $2\cdot 10^{-6}$ & $3.054\cdot 10^{-6}$\\
  Lazy-Increase & $1.1\cdot 10^{-6}$ & $2.22\cdot 10^{-6}$ & $3.492\cdot 10^{-6}$\\
  Pennant & $2.2\cdot 10^{-6}$ & $3.47\cdot 10^{-6}$ & $4.524\cdot 10^{-6}$\\
  Weak & $2.1\cdot 10^{-6}$ & $3.1\cdot 10^{-6}$ & $3.926\cdot 10^{-6}$

\end{tabular}
\end{table}

\begin{table}[h!]
\centering
\caption{Time Increase 2}
\begin{tabular}{l|lll}
& 10'000 & 100'000 & 500'000 \\
\hline
Strict & $10^{-7}$ & $1.1\cdot 10^{-7}$ & $1.28\cdot 10^{-7}$\\
Lazy-Insert & $3\cdot 10^{-7}$ & $4.3\cdot 10^{-7}$ & $4.54\cdot 10^{-7}$\\
Lazy-Increase & $4\cdot 10^{-7}$ & $3.9\cdot 10^{-7}$ & $4.06\cdot 10^{-7}$\\
Pennant & $1.3\cdot 10^{-6}$ & $1.37\cdot 10^{-6}$ & $1.562\cdot 10^{-6}$\\
Weak & $1.4\cdot 10^{-6}$ & $1.73\cdot 10^{-6}$ & $1.858\cdot 10^{-6}$

\end{tabular}
\end{table}

\begin{table}[h!]
\centering
\caption{Time Increase 3}
\begin{tabular}{l|lll}
\label{tab:time.last}
& 10'000 & 100'000 & 500'000 \\
\hline
Strict & $10^{-7}$ & $1.8\cdot 10^{-7}$ & $2.08\cdot 10^{-7}$\\
Lazy-Insert & $10^{-7}$ & $2.9\cdot 10^{-7}$ & $3.22\cdot 10^{-7}$\\
Lazy-Increase & $2\cdot 10^{-7}$ & $2\cdot 10^{-7}$ & $2.4\cdot 10^{-7}$\\
Pennant & $10^{-7}$ & $1.9\cdot 10^{-7}$ & $2.3\cdot 10^{-7}$\\
Weak & $3\cdot 10^{-7}$ & $2\cdot 10^{-7}$ & $2.3\cdot 10^{-7}$


\end{tabular}
\end{table}


\clearpage


\section{Source Code}
\subsection{increase2-comp.c++}
\lstinputlisting{../src/CPHSTL/Source/Priority-queue-framework/Benchmark/increase2-comp.c++}

\subsection{increase2-time.c++}
\lstinputlisting{../src/CPHSTL/Source/Priority-queue-framework/Benchmark/increase2-time.c++}

\subsection{increase3-comp.c++}
\lstinputlisting{../src/CPHSTL/Source/Priority-queue-framework/Benchmark/increase3-comp.c++}

\subsection{increase3-time.c++}
\lstinputlisting{../src/CPHSTL/Source/Priority-queue-framework/Benchmark/increase3-time.c++}

\subsection{pairing-heap-framework.h++}
\lstinputlisting{../src/v1/pairing-heap-framework.h++}

\subsection{pairing-heap-framework.i++}
\lstinputlisting{../src/v1/pairing-heap-framework.i++}

\subsection{pairing-heap-node.c++}
\lstinputlisting{../src/v1/pairing-heap-node.c++}

\subsection{pairing-heap-policy-strict.c++}
\lstinputlisting{../src/v1/pairing-heap-policy-strict.c++}

\subsection{pairing-heap-policy-lazy-insert.c++}
\lstinputlisting{../src/v1/pairing-heap-policy-lazy-insert.c++}

\subsection{pairing-heap-policy-lazy-increase.c++}
\lstinputlisting{../src/v1/pairing-heap-policy-lazy-increase.c++}


\end{document}

